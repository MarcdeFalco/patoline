
%                                              ************************************
%                                              LaTeX macros file - October 31, 2009
%                                              ************************************

%%%%%%%%%%%%%%%%%%%%%%%%%%%%%%
% English maths environments %
%%%%%%%%%%%%%%%%%%%%%%%%%%%%%%
\theoremstyle{plain}
\newtheorem{theorem}{Theorem}[section]
\newtheorem*{theorem*}{Theorem} % no number is given
\newtheorem*{maintheorem}{Main Theorem}
\newtheorem*{theoremA}{Theorem A}
\newtheorem*{theoremB}{Theorem B}
\newtheorem*{theoremC}{Theorem C}
\newtheorem{lemma}{Lemma}[section]
\newtheorem*{lemma*}{Lemma} % no number is given
\newtheorem{corollary}{Corollary}[section]
\newtheorem*{corollary*}{Corollary} % no number is given
\newtheorem{consequence}{Consequence}[section]
\newtheorem*{consequence*}{Consequence} % no number is given
\newtheorem{proposition}{Proposition}[section]
\newtheorem*{proposition*}{Proposition} % no number is given
\newtheorem{conjecture}{Conjecture}[section]
\newtheorem*{conjecture*}{Conjecture} % no number is given

\theoremstyle{definition}
\newtheorem{definition}{Definition}[section]
\newtheorem*{definition*}{Definition} % no number is given
\newtheorem{remark}{Remark}[section]
\newtheorem*{remark*}{Remark} % (single remark) no number is given
\newtheorem*{remarks*}{Remarks} % (list of remarks) no number is given
\newtheorem*{notation}{Notation} % no number is given
\newtheorem*{notations}{Notations} % no number is given
\newtheorem{question}{Question}[section]
\newtheorem*{question*}{Question} % (single question) no number is given
\newtheorem*{questions*}{Questions} % (list of questions) no number is given
\newtheorem{example}{Example}[section]
\newtheorem*{example*}{Example} % (single example) no number is given
\newtheorem*{examples*}{Examples} % (list of examples) no number is given
\newtheorem{exercise}{Exercise}[section]
\newtheorem*{exercise*}{Exercise} % (single exercise) no number is given
\newtheorem*{exercises*}{Exercises} % (list of exercises) no number is given
% \newtheorem{figure}{Figure}[section]


%%%%%%%%%%%%%%%%%%%%%%%%%%%%%
% French maths environments %
%%%%%%%%%%%%%%%%%%%%%%%%%%%%%
\theoremstyle{plain}
\newtheorem{thm}{Th�or�me}[section]
\newtheorem*{thm*}{Th�or�me} % no number is given
\newtheorem*{thmbase}{Th�or�me principal}
\newtheorem*{thmA}{Th�or�me A}
\newtheorem*{thmB}{Th�or�me B}
\newtheorem*{thmC}{Th�or�me C}
\newtheorem{lemme}{Lemme}[section]
\newtheorem*{lemme*}{Lemme} % no number is given
\newtheorem{cor}{Corollaire}[section]
\newtheorem*{cor*}{Corollaire} % no number is given
\newtheorem{csq}{Cons�quence}[section]
\newtheorem*{csq*}{Cons�quence} % no number is given
\newtheorem{prop}{Proposition}[section]
\newtheorem*{prop*}{Proposition} % no number is given
\newtheorem{conj}{Conjecture}[section]
\newtheorem*{conj*}{Conjecture} % no number is given

\theoremstyle{definition}
\newtheorem{d�f}{D�finition}[section]
\newtheorem*{d�f*}{D�finition} % no number is given
\newtheorem{rem}{Remarque}[section]
\newtheorem*{rem*}{Remarque} % (single remark) no number is given
\newtheorem*{rems*}{Remarques} % (list of remarks) no number is given
\newtheorem{ex}{Exemple}[section]
\newtheorem*{ex*}{Exemple} % (single example) no number is given
\newtheorem*{exs*}{Exemples} % (list of examples) no number is given
\newtheorem{exo}{Exercice}[section]
\newtheorem*{exo*}{Exercice} % (single exercise) no number is given
\newtheorem*{exos*}{Exercices} % (list of exercises) no number is given

\theoremstyle{remark}
\newtheorem*{preuve}{Preuve} % \qed has to be added at the end of the proof
\newtheorem*{d�mo}{D�monstration} % \qed has to be added at the end of the proof


%%%%%%%%%%%%%%%%%%%%%%%%
% Environment commands %
%%%%%%%%%%%%%%%%%%%%%%%%
\newcommand{\disp}{\displaystyle}
\newcommand{\hs}{\hspace{\stretch{1}}}
\newcommand{\vs}{\vspace{\stretch{1}}}

\renewcommand{\.}{{}_{\!}} 
% on the other hand, {}_{\,} gives almost the same result as \, (so, there is no need for a new command)


%%%%%%%%%%%%%%%%%
% Greek letters %
%%%%%%%%%%%%%%%%%
\renewcommand{\a}{\alpha}
\renewcommand{\b}{\beta}
\renewcommand{\c}{\gamma}
\newcommand{\C}{\Gamma}
\renewcommand{\d}{\delta}
\newcommand{\D}{\Delta}
\newcommand{\e}{\varepsilon}
\newcommand{\f}{\varphi}
\newcommand{\F}{\Phi}
\renewcommand{\i}{\iota}
\renewcommand{\k}{\kappa}
\renewcommand{\l}{\lambda}
\renewcommand{\L}{\Lambda}
\newcommand{\m}{\mu}
\newcommand{\n}{\nu}
\newcommand{\om}{\omega}
\newcommand{\Om}{\Omega}
\newcommand{\p}{\psi}
\renewcommand{\P}{\Psi}
\renewcommand{\r}{\rho}
\newcommand{\s}{\sigma}
\renewcommand{\S}{\Sigma}
\renewcommand{\t}{\theta}
\newcommand{\T}{\Theta}
\newcommand{\ups}{\upsilon} % \up is used by the package french in babel!
\newcommand{\Ups}{\Upsilon} % \U stands for the unitary group (see below)
\newcommand{\x}{\xi}
\newcommand{\X}{\Xi}
\newcommand{\z}{\zeta}


%%%%%%%%%%%%%%%%%%%%%%%%
% Calligraphic letters %
%%%%%%%%%%%%%%%%%%%%%%%%
\newcommand{\cA}{\mathcal{A}}
\newcommand{\cB}{\mathcal{B}}
\newcommand{\cC}{\mathcal{C}}
\newcommand{\cD}{\mathcal{D}}
\newcommand{\cE}{\mathcal{E}}
\newcommand{\cF}{\mathcal{F}}
\newcommand{\cG}{\mathcal{G}}
\newcommand{\cH}{\mathcal{H}}
\newcommand{\cI}{\mathcal{I}}
\newcommand{\cJ}{\mathcal{J}}
\newcommand{\cK}{\mathcal{K}}
\newcommand{\cL}{\mathcal{L}}
\newcommand{\cM}{\mathcal{M}}
\newcommand{\cN}{\mathcal{N}}
\newcommand{\cO}{\mathcal{O}}
\newcommand{\cP}{\mathcal{P}}
\newcommand{\cQ}{\mathcal{Q}}
\newcommand{\cR}{\mathcal{R}}
\newcommand{\cS}{\mathcal{S}}
\newcommand{\cT}{\mathcal{T}}
\newcommand{\cU}{\mathcal{U}}
\newcommand{\cV}{\mathcal{V}}
\newcommand{\cW}{\mathcal{W}}
\newcommand{\cX}{\mathcal{X}}
\newcommand{\cY}{\mathcal{Y}}
\newcommand{\cZ}{\mathcal{Z}}


%%%%%%%%%%%%%%%%%%
% Gothic letters %
%%%%%%%%%%%%%%%%%%
\newcommand{\gA}{\mathfrak{A}}
\newcommand{\gB}{\mathfrak{B}}
\newcommand{\gC}{\mathfrak{C}}
\newcommand{\gD}{\mathfrak{D}}
\newcommand{\gE}{\mathfrak{E}}
\newcommand{\gF}{\mathfrak{F}}
\newcommand{\gG}{\mathfrak{G}}
\newcommand{\gH}{\mathfrak{H}}
\newcommand{\gI}{\mathfrak{I}}
\newcommand{\gJ}{\mathfrak{J}}
\newcommand{\gK}{\mathfrak{K}}
\newcommand{\gL}{\mathfrak{L}}
\newcommand{\gM}{\mathfrak{M}}
\newcommand{\gN}{\mathfrak{N}}
\newcommand{\gO}{\mathfrak{O}}
\newcommand{\gP}{\mathfrak{P}}
\newcommand{\gQ}{\mathfrak{Q}}
\newcommand{\gR}{\mathfrak{R}}
\newcommand{\gS}{\mathfrak{S}}
\newcommand{\gT}{\mathfrak{T}}
\newcommand{\gU}{\mathfrak{U}}
\newcommand{\gV}{\mathfrak{V}}
\newcommand{\gW}{\mathfrak{W}}
\newcommand{\gX}{\mathfrak{X}}
\newcommand{\gY}{\mathfrak{Y}}
\newcommand{\gZ}{\mathfrak{Z}}


%%%%%%%%%
% Logic %
%%%%%%%%%
\newcommand{\as}{\ \mbox{\raisebox{.085ex}{$:$}\!$=$} \ } % direct assignment
\newcommand{\sa}{\ \mbox{$=$\!\raisebox{.085ex}{$:$}} \ } % reverse assignment
\newcommand{\imp}{\Longrightarrow} % implication
\newcommand{\con}{\Longleftarrow} % converse
% \newcommand{\eq}{\Longleftrightarrow} % equivalence >>> use \iff

% >>> width test: $| connector |$ (example: $| \as |$)

\newcommand{\st}{~|~} % such that
\newcommand{\St}{~\Big|~} % big such that

\newcommand{\ie}{\emph{i.\,e.}} % id est (= that is)
\newcommand{\eg}{\emph{e.\,g.}} % exempli gratia (= for example)


%%%%%%%%%%%%%%%%
% General sets %
%%%%%%%%%%%%%%%%
\newcommand{\Mor}[2]{\mathrm{Mor} \! \left( #1 , #2 \right)} 
% category morphisms from object #1 to object #2
% (>>> replace the argument #1 by \. #1 if #1 is denoted by a capital letter)
\newcommand{\End}[1]{\mathrm{End} \! \left( #1 \right)} % category endomorphisms of object #1
\newcommand{\Aut}[1]{\mathrm{Aut} \! \left( #1 \right)} % category automorphisms of object #1 (= group)

\newcommand{\Oset}{\mbox{\large $\varnothing$}} % empty set
\newcommand{\llist}[3]{#1_{#2} , \ldots , #1_{#3}} % line list = list of elements
\newcommand{\lset}[3]{\left\{ #1_{#2} , \ldots , #1_{#3} \right\}} % line set = set given by a list

\newcommand{\inc}{\subseteq} % inclusion
\newcommand{\incneq}{\subsetneq} % strict inclusion
% \newcommand{\setmin}{\raisebox{2.087335pt}{\scriptsize $\boldsymbol{\setminus}$}}
\newcommand{\setmin}{\raisebox{0.45ex}{\scriptsize $\smallsetminus$}} % difference between sets
\newcommand{\cart}{\! \times \!} % Cartesian product

\renewcommand{\to}{\longrightarrow} % [f : E \to F]
\newcommand{\map}[5]{\begin{array}{rcl} #1 #2 & \to & #3 \\ #4 & \longmapsto & #5 \end{array}} 
% map (or mapping)
\newcommand{\gr}[1]{\mathrm{Graph} \! \left( #1 \right)} % graph of the map #1
\newcommand{\I}[1]{\mathrm{I}_{\scriptscriptstyle #1}} % identity map of the set #1
\newcommand{\rest}[2]{#1_{\mathbf{|} #2}} % restriction of the map #1 to the subset #2
\newcommand{\1}[1]{\mathbf{1}_{\scriptscriptstyle \! #1}} % characteristic function of the subset #1

\newcommand{\act}{\! \cdot \!} % action operator

\renewcommand{\fam}[3]{\left( #1_{#2} \right)_{\! #2 \in #3}} 
% family where the parameter #2 ranges in the set #3

\newcommand{\flowR}[2]{\left( #1^{#2} \right)_{\! #2 \in \RR}} % flow where the parameter #2 ranges in \RR
\newcommand{\flow}[3]{\left( #1^{#2} \right)_{\! #2 \in #3}} 
% flow where the parameter #2 ranges in the (semi-)group #3

\newcommand{\seqN}[2]{\left( #1_{#2} \right)_{\! #2 \in \NN}} % sequence where the index #2 ranges in \NN
\newcommand{\seq}[3]{\left( #1 \right)_{\! #2 \geq #3}} % sequence where the index #2 ranges in [#3 , +\infty)

\newcommand{\serN}[2]{\sum_{#2 \geq 0} #1_{#2}} % series where the index #2 ranges in \NN
\newcommand{\ser}[3]{\sum_{#2 \geq #3} #1} % series where the index #2 ranges in [#3 , +\infty)

\newcommand{\TserN}[2]{\sum_{#2 \geq 0} #1_{#2} z^{#2}} % Taylor series where the index #2 ranges in \NN
\newcommand{\Tser}[3]{\sum_{#2 \geq #3} #1 z^{#2}} % Taylor series where the index #2 ranges in [#3 , +\infty)

\newcommand{\card}[1]{\mathrm{card} \! \left( #1 \right)} % cardinal


%%%%%%%%%%%%%%%
% Number sets %
%%%%%%%%%%%%%%%
\newcommand{\NN}{\mathbf{N}} % natural integers
\newcommand{\ZZ}{\mathbf{Z}} % integers
\newcommand{\DD}{\mathbf{D}} % decimal numbers
\newcommand{\QQ}{\mathbf{Q}} % rational numbers
\newcommand{\RR}{\mathbf{R}} % real numbers
\newcommand{\CC}{\mathbf{C}} % complex numbers
\newcommand{\HH}{\mathbf{H}} % quaternions


%%%%%%%%%%%%%%%%%%
% Geometric sets %
%%%%%%%%%%%%%%%%%%
\newcommand{\Zn}[1]{\mathbf{Z}^{#1 \!}} % standard lattice in R^{#1}
\newcommand{\Qn}[1]{\mathbf{Q}^{#1 \!}} % 
\newcommand{\Rn}[1]{\mathbf{R}^{\! #1 \!}} % #1-dimensional real Cartesian space
\newcommand{\Bn}[1]{\mathbf{B}^{#1 \!}} % canonical Euclidean unit open ball in R^{#1}
\newcommand{\Sn}[1]{\mathbf{S}^{#1 \!}} % canonical Euclidean unit sphere in R^{#1 + 1}
\newcommand{\Tn}[1]{\mathbf{T}^{#1 \!}} % #1-dimensional real canonical torus (= R^{#1} / Z^{#1})
\newcommand{\Cn}[1]{\mathbf{C}^{#1 \!}} % #1-dimensional complex Cartesian space
\newcommand{\Hn}[1]{\mathbf{H}^{#1 \!}} % #1-dimensional real hyperbolic space
\newcommand{\Pn}[2]{\mathbf{P}^{#1 \!}(#2)} % #1-dimensional projective space over the field #2 (= P(K^{#1 + 1}))


%%%%%%%%%%%%
% Topology %
%%%%%%%%%%%%
\newcommand{\intr}[1]{\overset{\; _{\circ}}{#1}} % interior
\newcommand{\clos}[1]{\overline{#1}} % closure
\newcommand{\bd}[1]{\partial #1} % boundary (closure less interior)


%%%%%%%%%%%%
% Analysis %
%%%%%%%%%%%%
\renewcommand{\leq}{\leqslant} % less or equal
\renewcommand{\geq}{\geqslant} % greater or equal

\newcommand{\epi}[1]{\mathrm{Epi} \! \left( #1 \right)} % epigraph of the function #1

\DeclareMathOperator{\acosh}{acosh} % inverse hyperbolic cosine
\DeclareMathOperator{\asinh}{asinh} % inverse hyperbolic sine
\DeclareMathOperator{\atanh}{atanh} % inverse hyperbolic tangent

\newcommand{\logb}[2][10]{\log_{#1}{\! \left( #2 \right)}} 
% logarithm of #2 to the base #1 (10 is the default value)

\let\oldint\int
\renewcommand{\int}[4]{\oldint_{\! #1}^{#2} \!\!\!\! #3 \mathrm{d} #4} % integral
% on ne peut pas red�finir \int en utilisant l'ancien \int, m�me en passant par une commande interm�diaire
% (c'est pourquoi, on utilise \let\oldint\int, l'ancien \int �tant renomm� \oldint)

\newcommand{\dev}[3]{\left[ {}^{\stackrel{{}^{\stackrel{}{}}}{}} 
{#1}^{\stackrel{{}^{\stackrel{}{}}}{}} \right]_{\! #2}^{\! #3}} 
% difference of the function #1 evaluated at #2 and #3 (= #1(#3) - #1(#2))

\renewcommand{\iint}[4]{\oldint \!\!\!\! \oldint_{\! #1} \!\! #2 \, \mathrm{d} #3 \mathrm{d} #4} 
% double integral
\renewcommand{\iiint}[5]{\oldint \!\!\!\! \oldint \!\!\!\! \oldint_{\! #1} \!\! #2 \, 
\mathrm{d} #3 \mathrm{d} #4 \mathrm{d} #5} 
% triple integral

\newcommand{\BB}{\mathrm{B}} % B(X , Y) = bounded functions from X (set) to Y (metric space)
\renewcommand{\ll}[1]{\ell^{#1 \!}} % l^{#1} property of a family (summability)
\newcommand{\LL}[1]{\mathrm{L}^{\!\. \mbox{\raisebox{.12ex}{\tiny $#1$}} \!}} 
% L^{#1} property of a function with respect to Lebesgue integration

\newcommand{\goes}{\rightarrow} % convergence arrow
\newcommand{\we}{\mathop{\mbox{\large $\rightharpoonup$}}} 
\newcommand{\weak}{\we_{n \goes +\infty}} % weak convergence arrow [f_{n} \weak f]

\newcommand{\negligible}{\mathop{\textsf{o}}} 
% \mathop d�finit un op�rateur du type \sum permettant les indices SOUS l'op�rateur en mode \displaystyle
\renewcommand{\neg}[2]{\negligible_{^{#1}} \! \left( #2 \right)} % negligible to the function #2 at the point #1
\newcommand{\dominated}{\mathop{\textsf{O}}}
\newcommand{\dom}[2]{\dominated_{^{#1}} \! \left( #2 \right)} % dominated by the function #2 at the point #1
\newcommand{\equivalent}{\mathop{\thicksim}}
\newcommand{\equ}[2]{\equivalent_{^{#1}} #2} % equivalent to the function #2 at the point #1
\newcommand{\DL}[2]{\mathrm{DL}_{#1 \,} \!\! \left( #2 \right)} 
% finite Taylor expansion of order #1 at the point #2

\newcommand{\DSE}[1]{\mathrm{DSE} \! \left( #1 \right)} % infinite Taylor expansion at the point #1

\newcommand{\E}[1]{\mathrm{E} \! \left( #1 \right)} % partie enti�re (French for integer part: [3.1415] = 3)

\newcommand{\Res}[2]{\mathrm{Res} \! \left( #1 , #2 \right)} 
% residue of the meromorphic function #1 at the singularity #2

% >>> use \Cl{\,}(\. E , F) or \Cl{0}(\. E , F) for the set of continuous maps from E to F

\newcommand{\Homeo}[1]{\mathrm{Homeo} \! \left( #1 \right)} % homeomorphism group of #1


%%%%%%%%%%%%%%%%%%%%%%%%%%%%
% Probability & Statistics %
%%%%%%%%%%%%%%%%%%%%%%%%%%%%
% \newcommand{\PP}[1]{\mathbf{P} \! \left( #1 \right)} 
% probability of the event #1 (already defined in geometry)
\newcommand{\EE}[1]{\mathbf{E} \! \left( #1 \right)} % expected value of the random variable #1
\newcommand{\var}[1]{\mathrm{Var} \! \left( #1 \right)} % variance of the random variable #1
\newcommand{\sd}[1]{\boldsymbol{\sigma} \! \left( #1 \right)} 
% standard deviation (�cart-type in French) of the random variable #1
\newcommand{\cov}[2]{\mathrm{Cov} \! \left( #1 , #2 \right)} 
% covariance of the pair of random variables (#1 , #2)


%%%%%%%%%%%%%%%%%%%
% Complex numbers %
%%%%%%%%%%%%%%%%%%%
\renewcommand{\Re}[1]{\mathrm{Re} \! \left( #1 \right)} % real part
% \renewcommand{\Im}[1]{\mathrm{Im} \! \left( #1 \right)} % imaginary part (already defined in linear algebra)
\renewcommand{\bar}{\overline} % conjugate


%%%%%%%%%%%%%%%%%%
% Linear algebra %
%%%%%%%%%%%%%%%%%%
\renewcommand{\vec}[1]{\overset{\rightarrow \!}{#1}} % small vector arrow
\newcommand{\vect}[1]{\overrightarrow{#1}} % long vector arrow
\newcommand{\lvec}[3]{\left( #1_{#2} , \ldots , #1_{#3} \right)} % line vector
\newcommand{\cvec}[3]{{\mbox{\scriptsize $\begin{pmatrix} #1_{#2} \\ \vdots \\ #1_{#3} \end{pmatrix}$}}} 
% column vector

\newcommand{\Mat}{\mathrm{Mat}} % Mat(f , B , B')

\newcommand{\Trans}[2]{\mathrm{Trans} \! \left( #1 , #2 \right)} 
% (English) transformation matrix from basis #1 to basis #2
\newcommand{\Pass}[2]{\mathrm{Pass} \! \left( #1 , #2 \right)} 
% (Fran�ais) matrice de passage de la base #1 � la base #2

\newcommand{\Ker}[1]{\mathrm{Ker} \! \left( #1 \right)} % kernel
\newcommand{\Coker}[1]{\mathrm{Coker} \! \left( #1 \right)} % cokernel
\renewcommand{\Im}[1]{\mathrm{Im} \! \left( #1 \right)} % image

\renewcommand{\dim}[1]{\mathrm{dim} \! \left( #1 \right)} % dimension (without specifying the field)
\newcommand{\codim}[2]{\mathrm{codim}_{#1 \,} \!\! \left( #2 \right)} % codimension of #2 in #1
\newcommand{\dimk}[2]{\mathrm{dim}_{#1 \,} \!\! \left( #2 \right)} % dimension of #2 over the field #1
\newcommand{\rk}[1]{\mathrm{rk} \! \left( #1 \right)} % rank
\newcommand{\rg}[1]{\mathrm{rg} \! \left( #1 \right)} % rang (French for rank)
\newcommand{\sign}[1]{\mathrm{sign} \! \left( #1 \right)} % signature of a quadratic form
 
\newcommand{\Vect}[1]{\mathrm{Vect} \! \left( #1 \right)} % spanned vector subspace

\newcommand{\tp}[1]{{}^{\disp \stackrel{\mathrm{t}}{}} \! #1} % transpose

\newcommand{\tr}[1]{\mathrm{tr} \! \left( #1 \right)} % trace

\let\olddet\det
\renewcommand{\det}[1]{\olddet{\! \left( #1 \right)}} % determinant of an endomorphism or a matrix
\newcommand{\detb}[2]{\olddet\nolimits_{\! #1 \,}{\!\! #2}} 
% determinant in the basis #1 of the family of vectors #2

\newcommand{\spe}[2]{\sigma_{\! #1 \,} \!\! \left( #2 \right)} % spectrum of #2 over the field #1
% >>> use \s(T) for the spectrum of the linear operator T (without specifying the field)

% >>> use \LL{}(E , F) for the set of linear maps from E to F % \L stands for capital lambda (see above)

\newcommand{\A}{\mathrm{A}} % set of affine maps = A(E , F)
\newcommand{\M}{\mathrm{M}} % set of n x p matrices = M(n x p , A) (or M(n , A) if n = p), where A is a ring

\newcommand{\GL}{\mathrm{GL}} % linear group = GL(n , A) or GL(E)
\newcommand{\GA}{\mathrm{GA}} % affine group = GA(n , A) or GA(E)
\newcommand{\PGL}{\mathrm{PGL}} % projective linear group = PGL(n , A) or PGL(E)
\newcommand{\SL}{\mathrm{SL}} % special linear group = SL(n , A) or SL(E)
\newcommand{\PSL}{\mathrm{PSL}} % projective special linear group = PSL(n , A) or PSL(E)
\renewcommand{\O}{\mathrm{O}} % orthogonal group = O(n , R) or O(E) (E Euclidean vector space)
\newcommand{\SO}{\mathrm{SO}} % special orthogonal group = SO(n , R) or SO(E) (E Euclidean vector space)
\newcommand{\GO}{\mathrm{GO}} % group of vector similarities = GO(n , A) or GO(E)
\newcommand{\U}{\mathrm{U}} % unitary group = U(n , C) or U(E) (E Hermitian vector space)
\newcommand{\SU}{\mathrm{SU}} % special unitary group = SU(n , C) or SU(E) (E Hermitian vector space)
\newcommand{\Sp}{\mathrm{Sp}} % symplectic group = Sp(2n , R) or Sp(E)

\renewcommand{\deg}[1]{\mathrm{d}^{\circ} \! \left( #1 \right)} % degree of a polynomial


%%%%%%%%%%%%
% Geometry %
%%%%%%%%%%%%
\newcommand{\paral}{\diagup \!\!\! \diagup} % parallel
\newcommand{\nparal}{\diagup \!\!\! \diagup \hspace{-3.75mm} \backslash \,\,} % non-parallel

\newcommand{\Aff}[1]{\mathrm{Aff} \! \left( #1 \right)} % spanned affine subspace
\newcommand{\Bary}[1]{\mathrm{Bar} \! \left( #1 \right)} % barycenter
\newcommand{\Conv}[1]{\mathrm{Conv} \! \left( #1 \right)} % convex hull

\newcommand{\PP}[1]{\mathbf{P} \! \left( #1 \right)} % projective space of the vector space #1
\newcommand{\Proj}[1]{\mathrm{Proj} \! \left( #1 \right)} % spanned projective subspace

\newcommand{\norm}[1]{\left\| #1 \right\|} % norm
\newcommand{\onenorm}[1]{\left\| #1 \right\|_{_{1}}} % L1-norm
\newcommand{\twonorm}[1]{\left\| #1 \right\|_{_{2}}} % L2-norm
\newcommand{\pnorm}[1]{\left\| #1 \right\|_{_{p}}} % Lp-norm
\newcommand{\supnorm}[1]{\left\| #1 \right\|_{_{\infty}}} % sup-norm (or uniform norm)
\newcommand{\opnorm}[1]{\left| \! \left| \! \left| #1 \right| \! \right| \! \right|} 
% operator norm

\newcommand{\<}{\langle} % left bracket
\renewcommand{\>}{\rangle} % right bracket
\newcommand{\scal}[2]{\left\langle #1 , #2 \right\rangle} % scalar product

\newcommand{\gmeas}[2]{\sphericalangle (#1 , #2)} 
% geometric measure (or spherical distance) of the pair of vectors (#1 , #2) 
% (wrongly called geometric 'angle')
\newcommand{\oangle}[2]{\widehat{(#1 , #2)}} % oriented angle
\newcommand{\noangle}[2]{\bar{(#1 , #2)}} % non-oriented angle
\newcommand{\meas}{\mathrm{meas}} % (English) measure in R/2piZ (R/piZ) of oriented (non-oriented) angles
\newcommand{\mes}{\mathrm{mes}} % (Fran�ais) mesure des angles orient�s (non orient�s) dans R/2piZ (R/piZ)

\newcommand{\diam}{\mathrm{diam}} % diameter
\newcommand{\vol}{\mathrm{vol}} % volume


%%%%%%%%%%%%%%%%%%%%%%%%%
% Differential geometry %
%%%%%%%%%%%%%%%%%%%%%%%%%

\newcommand{\der}[2]{\frac{\mathrm{d} #1}{\mathrm{d} #2}} % derivative (Leibniz notation)
\newcommand{\dern}[3]{\frac{\mathrm{d}^{#3} #1}{\mathrm{d} #2^{#3}}} % nth derivative (Leibniz notation)
\newcommand{\Tg}[3]{T_{\! #1} #2 \! \cdot \! #3} % tangent linear map

\newcommand{\pder}[2]{\frac{\partial #1}{\partial #2}} % partial derivative
\newcommand{\ppder}[3]{\frac{\partial^{2} #1}{\partial #2 \partial #3}} % second order partial derivative
\newcommand{\pppder}[4]{\frac{\partial^{3} #1}{\partial #2 \partial #3 \partial #4}} 
% third order partial derivative
\newcommand{\pnder}[3]{\frac{\partial^{#3} #1}{\partial #2_{1} \cdots \partial #2_{#3}}} 
% partial derivative of order #3

\newcommand{\ed}[1]{\mathrm{d} #1} % exterior differential

\newcommand{\J}[1]{\mathrm{J} \! \left( #1 \right)} % Jacobian matrix of the map #1 (as a map)
\newcommand{\Jac}[1]{\mathrm{Jac} \! \left( #1 \right)} 
% Jacobian determinant (or Jacobian) of the map #1 (as a map)
\renewcommand{\H}[1]{\mathrm{H} \! \left( #1 \right)} % Hessian matrix of the map #1 (as a map)
\newcommand{\Hes}[1]{\mathrm{Hes} \! \left( #1 \right)} 
% Hessian determinant (or Hessian) of the map #1 (as a map)

\newcommand{\grad}[1]{\mathrm{grad} \! \left( #1 \right)} % gradient
\newcommand{\curl}[1]{\mathrm{curl} \! \left( #1 \right)} % curl
\newcommand{\rot}[1]{\mathrm{rot} \! \left( #1 \right)} % rotationnel (French for curl)
\renewcommand{\div}[1]{\mathrm{div} \! \left( #1 \right)} % divergence

\newcommand{\Cl}[1]{\mathrm{C}^{#1 \!}} % class C^{#1}

\newcommand{\Diff}[2]{\mathrm{Diff}^{#1 \!} \! \left( #2 \right)} 
% group of C^#1 diffeomorphisms of the manifold #2
\newcommand{\Symp}[1]{\mathrm{Symp} \! \left( #1 \right)} 
% group of smooth symplectomorphisms of the manifold #1
\newcommand{\Cont}[1]{\mathrm{Cont} \! \left( #1 \right)} 
% group of smooth contactomorphisms of the manifold #1









% R�GLES TYPOGRAPHIQUES : 
% x + y (avec blancs) , x y (avec blanc) , x \times y (avec blancs) , x / y (avec blancs).

